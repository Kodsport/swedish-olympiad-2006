\problemname{Ramsan}


{\em
Ole Dole Doff\\ 
Kinke Lane Koff\\ 
Koffe Lane, Binke Bane\\ 
Ole Dole Doff \\
Ärtan Pärtan Piff\\
Ärtan Pärtan Paff \\
Similimaka, Kuckelikaka\\ 
Ärtan Pärtan Poff
}

Ramsor av detta slag har under alla tider använts av barn för att välja ut någon slumpvis, till
exempel vem som ska stå då man leker kurragömma. Barnen bildar en ring, där den som ``räknar''
för varje ord i ramsan flyttar fingret, från ett barn till nästa, runt ringen. När ramsan
tar slut får det utpekade barnet lämna ringen. Ramsan läses på nytt med början på det barn som
står omedelbart efter det uträknade. Denna procedur fortsätter tills endast ett barn återstår, det utvalda.

\section*{Indata}
Den första raden av innehåller de två heltalen $k$ och $n$ ($2 \leq k,n \leq 100)$, antalet ord
i ramsan och antalet barn i ringen. Barnen är numrerade från $1$ till $n$ åt samma håll som räkningen
sker. Nummer $1$ är det barn man börjar räkna på.

\section*{Utdata}
Skriv ut ett heltal: numret på det barn som blir sist kvar.

\section*{Poängsättning}
Din lösning kommer att testas på flera testfall. För att få 100 poäng så måste du klara alla testfall.

\section*{Förklaring av exempelfall}
I exempelfall $1$ pekar barn $1$ på följande barn i ordning: $1$, $2$, $3$, $1$. Barn $1$ åker därmed ut.
Eftersom barn $2$ är näst på led räknar den upp $2$, $3$, $2$, $3$. Då lämnar barn $3$ och barn $2$ är
det sista barnet kvar.

I exempelfall $2$ är det första barnet som åker ut nummer $7$ och därefter nummer $15$, $8$ och $2$.
